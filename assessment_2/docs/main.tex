\documentclass[a4paper, 12pt, twoside]{article}

% Packages

\begin{document}

\title{GEOG5790M: Ensemble Kalman Filter}
\author{Keiran Suchak}
\date{05/05/19}

\maketitle
\tableofcontents

\newpage
\section{Aim}\label{sec:aim}

This document has been produced as part of the second assessment for the
GEOG5790M module.
The aim of this module is to further develop Python programming skills.
With this aim in mind, the module focuses on the following processes:
\begin{itemize}
    \item Data processing,
    \item Analysis and visualisation, and
    \item Modelling.
\end{itemize}

As with the second assessment for the preceding introductory module, the scope
for this assessment is left relatively open-ended with the hope that something
of use can be produced.
The code that has been produced therefore aims to implement a data assimilation
technique known as the Ensemble Kalman Filter (EnKF) which can aid the
simulation process when new data is provided regarding the system that is being
modelled.

\subsection{Problem Specification}\label{sub:aim:problem}



\section{Implementation}\label{sec:implementation}

\section{Usage}\label{sec:usage}

\section{Future Improvements}\label{sec:improvements}

\newpage
\appendix
\section{Required Packages}\label{sec:requirements}

\end{document}
